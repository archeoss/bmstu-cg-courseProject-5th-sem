\chapter*{Введение}
\addcontentsline{toc}{chapter}{Введение}
Целью данного курсового проекта является разработка ПО, визуализирующие различные композиции тел вращения.
Это включает в себя такие вещи, как
\begin{enumerate}
    \item Воссоздание сложных композиций путем объеденения, пересечения и исключения различных примитивов
    \item Создание тел вращения путем вращения выше обозначенных композиций вокруг установленной оси
    \item Представление полноценной освещенной сцены с композициями объектов
    \item Обзор рабочей сцены из различных ее участков посредством нескольких камер
    \item Интерактивность с различными объектами сцены, такими как:
    \begin{enumerate}
        \item Камеры
        \item Модели
        \item Композиты
    \end{enumerate}
\end{enumerate}
Для достижения поставленных целей необходимо решить следующие задачи:
\begin{itemize}
    \item Описать структуру трехмерной сцены;
    \item Реализовать оптимальные алгоритмы представления, преобразования и визуализации твердотельной модели;
    \item Реализовать аппаратную обработку всех элментов сцены;
    \item Спроектировать процесс моделирования сцены;
    \item Описать использующиеся структуры данных;
    \item Определить средства программной реализации;
    \item Провести экспериментальные замеры временных характеристик разработанного ПО.
\end{itemize}
В ходе курсовой работы будут затронуты такие темы, как:
\begin{itemize}
    \item[$-$] Понятия о телах и поверхностях вращения
    \item[$-$] Удаление невидимых линий и поверхностей
    \item[$-$] Бинарные операции над телами
    \item[$-$] Методы закрашивания объемных тел
    \item[$-$] Преимущества и особенности языка Rust
\end{itemize}
