\chapter*{Введение}
\addcontentsline{toc}{chapter}{Введение}

На момент написания данной работы, технологии в области компьютерный графики сильно развились с момента начала их развития. Закон Мура гласит, что примерно каждые 2 года также возрастали в 2 раза производительные мощности ЭВМ. В частности же, на практике закон Мура был применим к широкому спектру компонентов: количество транзисторов на кристалле интегральной схемы, размер оперативной памяти, чувствительность сенсоров и даже количество и размер пикселей цифровых камер.

Хоть и данный закон с 2010 года больше не выполняется относительно количества транзисторов (их рост слегка замедлился), нельзя не отметить сильно возросшие мощности по сравнению с концом XX века

В связи с этим особенно важно обращать внимание на методы обработки изображения, их хранения и вывода на экран. Современный программист имеет сильный соблазн в максимальном использовании ресурсов своей рабочей машины, и в связи с этим им порождается крайне ресурсоемкий код

Во второе десятилетие XXI века крайне важно обращать не только на поставленные задачи, но и также на эффективность выбранных программистом методов

Целью данного курсового проекта является разработка ПО, визуализирующего трехмерные фрактальные поверхности.
Для достижения данной цели необходимо решить следующие задачи:
\begin{itemize}
    \item Описать структуру трехмерной сцены;
    \item Реализовать оптимальные алгоритмы представления, преобразования и визуализации твердотельной модели;
    \item Реализовать аппаратную обработку всех элментов сцены;
    \item Спроектировать процесс моделирования сцены;
    \item Описать использующиеся структуры данных;
    \item Определить средства программной реализации;
    \item Провести экспериментальные замеры временных характеристик разработанного ПО.
\end{itemize}
%    \item
%• описать структуру синтезируемой трехмерной сцены;
%• описать существующие алгоритмы построения реалистичных изображений;
%• выбрать и обосновать выбор реализуемых алгоритмов;
%• привести схемы реализуемых алгоритмов;
%• 
%• описать использующиеся структуры данных;
%• описать структуру разрабатываемого ПО;
%• 
%• 
%• 