\chapter{Технологический раздел}

\section{Средства реализации}

Основным языком программирования является мультипарадигменный язык Rust\cite{rust}.
Одно из главных достоинств данного языка это гарантия безопасной работы с памятью при помощи системы
статической проверки ссылок, так называемый Borrow Checker\cite{borrow-checker}.
Важно отметить, что язык программирования Rust сопоставим по скорости с такими языками как С и С++,
предоставляя в то же время более широкий функционал для тестирования и контроля памяти.

В качестве графического интерфейса была выбрана библиотека Slint\cite{slint}
(или иначе crate в контексте языка Rust) в связке с библиотекой tiny-skia\cite{tiny-skia}
для рендеринга изображения при помощи процессора.
Выбранная графическая библиотека предоставляет методы для работы с непосредственным пиксельным видеобуфером,
а также крайне гибкую настройку пользовательского интерфейса.
Помимо этого Slint дает инструментарий для запуска приложения в браузере при непосредственном участии WebAssembly
при практически нулевых затратах со стороны программиста.

Для тестирования ПО использовались инструменты Cargo\cite{cargo} - пакетного менеджера языка Rust,
поставляемого вместе с компилятором из официального источника.

Работа была проведена в среде разработки CLion\cite{clion} от компании JetBrains\cite{JB}
с дополнительным плагином "Rust" для поддержки синтаксиса языка.

\section{Реализация алгоритмов}

В листинге \ref{lst:zBuffer} представлена реализация Z-буффера на языке Rust.
В листинге \ref{lst:frame_model} представлена реализация проволочной фигуры.

\begin{lstlisting}[style=rust, label=lst:zBuffer, caption={Реализация алгоритма Z-буффера}]

    #[derive(Copy, Clone)]
pub struct Vector3D<T> {
    pub x: T,
    pub y: T,
    pub z: T,
}
impl<T> Vector3D<T> {
    pub fn new(x: T, y: T, z: T) -> Vector3D<T> {
        Vector3D {
            x: x,
            y: y,
            z: z,
        }
    }
}
impl<T: NumCast> Vector3D<T> {
    pub fn to<V: NumCast>(self) -> Vector3D<V> {
        Vector3D {
            x: NumCast::from(self.x).unwrap(),
            y: NumCast::from(self.y).unwrap(),
            z: NumCast::from(self.z).unwrap(),
        }
    }
}
impl Vector3D<f32> {
    pub fn norm(self) -> f32 {
        return (self.x*self.x+self.y*self.y+self.z*self.z).sqrt();
    }
    pub fn normalized(self, l: f32) -> Vector3D<f32> {
        return self*(l/self.norm());
    }
}
impl<T: fmt::Display> fmt::Display for Vector3D<T> {
    fn fmt(&self, f: &mut fmt::Formatter) -> fmt::Result {
        write!(f, "({},{},{})", self.x, self.y, self.z)
    }
}
impl<T: Add<Output = T>> Add for Vector3D<T> {
    type Output = Vector3D<T>;
    fn add(self, other: Vector3D<T>) -> Vector3D<T> {
        Vector3D { x: self.x + other.x, y: self.y + other.y, z:  self.z + other.z}
    }
}
impl<T: Sub<Output = T>> Sub for Vector3D<T> {
    type Output = Vector3D<T>;
    fn sub(self, other: Vector3D<T>) -> Vector3D<T> {
        Vector3D { x: self.x - other.x, y: self.y - other.y, z:  self.z - other.z}
    }
}
impl<T: Mul<Output = T> + Add<Output = T>> Mul for Vector3D<T> {
    type Output = T;
    fn mul(self, other: Vector3D<T>) -> T {
        return self.x*other.x + self.y*other.y + self.z*other.z;
    }
}
impl<T: Mul<Output = T> + Copy> Mul<T> for Vector3D<T> {
    type Output = Vector3D<T>;
    fn mul(self, other: T) -> Vector3D<T> {
        Vector3D { x: self.x * other, y: self.y * other, z:  self.z * other}
    }
}
impl<T: Mul<Output = T> + Sub<Output = T> + Copy> BitXor for Vector3D<T> {
    type Output = Vector3D<T>;
    fn bitxor(self, v: Vector3D<T>) -> Vector3D<T> {
        Vector3D { x: self.y*v.z-self.z*v.y, y: self.z*v.x-self.x*v.z, z: self.x*v.y-self.y*v.x}
    }
}
\end{lstlisting}

\begin{lstlisting}[style=rust, label=lst:frame_model, caption={Реализация проволочной модели}]
...
// includes
...
#[derive(Clone)]
pub struct FrameFigure
{
    points: Vec<Point>,
    edges: Vec<Edge>,
}

#[derive(Clone)]
pub struct FrameModel
{
    figure: Rc<RefCell<FrameFigure>>,
    transform: Matrix4<f32>,
}

impl FrameFigure
{
    pub fn new() -> FrameFigure
    {
        FrameFigure
        {
            points: Vec::new(),
            edges: Vec::new(),
        }
    }

    pub fn new_with_points(points: Vec<Point>) -> FrameFigure
    {
        FrameFigure
        {
            points,
            edges: Vec::new(),
        }
    }

    pub fn new_with_edges(edges: Vec<Edge>) -> FrameFigure
    {
        FrameFigure
        {
            points: Vec::new(),
            edges,
        }
    }

    pub fn new_with_points_and_edges(points: Vec<Point>, edges: Vec<Edge>) -> FrameFigure
    {
        FrameFigure
        {
            points,
            edges,
        }
    }

    pub fn get_points(&self) -> &Vec<Point>
    {
        &self.points
    }

    pub fn get_edges(&self) -> &Vec<Edge>
    {
        &self.edges
    }

    pub fn get_points_mut(&mut self) -> &mut Vec<Point>
    {
        &mut self.points
    }

    pub fn get_edges_mut(&mut self) -> &mut Vec<Edge>
    {
        &mut self.edges
    }

    pub fn add_point(&mut self, point: Point)
    {
        self.points.push(point);
    }

    pub fn add_edge(&mut self, edge: Edge)
    {
        self.edges.push(edge);
    }

    pub fn remove_point(&mut self, index: usize)
    {
        self.points.remove(index);
    }

    pub fn remove_edge(&mut self, index: usize)
    {
        self.edges.remove(index);
    }

    pub fn get_point(&self, index: usize) -> &Point
    {
        &self.points[index]
    }

    pub fn get_edge(&self, index: usize) -> &Edge
    {
        &self.edges[index]
    }

    pub fn get_point_mut(&mut self, index: usize) -> &mut Point
    {
        &mut self.points[index]
    }

    pub fn get_edge_mut(&mut self, index: usize) -> &mut Edge
    {
        &mut self.edges[index]
    }

    pub fn get_center(&self) -> Point
    {
        let mut max = self.points[0];
        let mut min = self.points[0];

        for point in &self.points
        {
            max = Point::new(max.get_x().max(point.get_x()), max.get_y().max(point.get_y()), max.get_z().max(point.get_z()));
            min = Point::new(min.get_x().min(point.get_x()), min.get_y().min(point.get_y()), min.get_z().min(point.get_z()));
        }

        (max + min) / Point::new(2.0, 2.0, 2.0)
    }
}

impl FrameModel
{
    pub(crate) fn new(figure: Rc<RefCell<FrameFigure>>) -> FrameModel
    {
        FrameModel
        {
            figure,
            transform: Matrix4::new(1.0, 0.0, 0.0, 0.0,
                                    0.0, 1.0, 0.0, 0.0,
                                    0.0, 0.0, 1.0, 0.0,
                                    0.0, 0.0, 0.0, 1.0),
        }
    }
}

impl Model for FrameModel
{
    type Output = FrameFigure;
    fn get_model(&self) -> Rc<RefCell<Self::Output>>
    {
        self.figure.clone()
    }

    fn get_center(&self) -> Point {
        self.figure.borrow().get_center()
    }

    fn get_transform(&self) -> Matrix4<f32>
    {
        self.transform
    }
    fn transform(&mut self, transform: Matrix4<f32>) {
        self.transform = self.transform * transform;
    }
}

impl Visibility for FrameModel
{
    fn is_visible(&self) -> bool
    {
        true
    }
}
\end{lstlisting}
\section*{Вывод}

В данном разделе были рассмотрены средства, с помощью которых было реализовано ПО, а также представлены листинги кода с реализацией алгоритмов компьютерной графики.


