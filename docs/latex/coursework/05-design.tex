\section{Конструкторский раздел}
\subsection{Требования к программному обеспечению}
Программа должна предоставлять доступ к функционалу:
\begin{itemize}
    \item Добавление, удаление объектов/композитов;
    \item Вращение, масштабирование, перемещение объектов/композитов;
    \item Редактирование параметров (ширина, высота) объектов;
    \item Изменение положения источника света;
    \item Редактирование объектов (закраска нужным цветом);
    \item Редактирование композитов;
    \item Передвижение по сцене (перемещение и вращение камеры);
\end{itemize}

К программе предъявляются следующие требования:

\begin{itemize}
    \item время отклика программы должно быть менее 1 секунды для корректной работы в интерактивном режиме;
    \item программа должна корректно реагировать на любые действия пользователя.
\end{itemize}

\subsection{Разработка алгоритмов}

\subsubsection{Алгоритм Z-буфера}
\begin{enumerate}
    \item Всем элементам буфера кадра присвоить фоновое значение
    \item Инициализировать Z буфер минимальными значениями глубины
    \item Выполнить растровую развертку каждого многоугольника сцены:
    \begin{itemize}
        \item[$-$] Для каждого пикселя, связанного с многоугольником вычислить его
        глубину z(x, y)
        \item[$-$] Сравнить глубину пискселя со значением, хранимым в Z буфере.

            Если \(z(x, y) > z\_buf(x, y) \), тогда\newline
            \(z\_buf(x,y) = z(x,y), color(x, y) = colorOfPixel.\)
    \end{itemize}
    \item Отобразить результат.
\end{enumerate}

На рис. \ref{img:z-buffer_algo} изображена схема алгоритма Z-буфера

\img{180mm}
{z-buffer_algo} % Имя файла без расширения (файл должен быть расположен в директории inc/img/)
{Алгоритм Z-буфера} % Подпись рисунка
\clearpage
%
%\subsubsection{Простой метод освещения}
%В простом методе освещения интенсивность рассчитывается по закону Ламберта:
%\[I = I0*cos(\alpha)\] где
%I – результирующая интенсивность света в точке
%I0 – интенсивность источника
%\(\alpha\) – угол между нормалью к поверхности и вектором направления света
\subsubsection{Модифицированный алгоритм, использующий z-буфер}
\begin{enumerate}
    \item Для каждого направленного источника света:
    \begin{enumerate}
        \item[$-$] Инициализировать теневой z-буфер минимальным значением глубины;
        \item[$-$] Определить теневой z-буфер для источника.
    \end{enumerate}
    \item Выполнить алгоритм z-буфера для точки наблюдения. При этом, если
    некоторая поверхность оказалась видимой относительно текущей точки
    наблюдения, то проверить, видима ли данная точка со стороны источников света
    \item Для каждого источника света:
    \begin{enumerate}
        \item[$-$] координаты рассматриваемой точки \((x, y, z)\) линейно преобразовать из вида наблюдателя в координаты
        \((x0, y0, z0)\) на виде из рассматриваемого источника света;
        \item[$-$] cравнить значение \(z\_shadowBuf (x0, y0)\) со значением \(z0(x0, y0)\).
        Если \(z0(x0, y0) < zshadowBuf (x0, y0)\), то пиксел высвечивается с учетом его
        затемнения, иначе точка высвечивается без затемнения
    \end{enumerate}
    \item Отобразить результат.
\end{enumerate}

\subsection{Выбор используемых типов и структур данных}
Для разрабатываемого ПО нужно будет реализовать следующие типы и
структуры данных.
\begin{enumerate}
\item \textbf{Источник света} – направленностью света.
\item \textbf{Сцена} – задается объектами сцены.
\item \textbf{Объекты сцены} – задаются вершинами и гранями.
\item \textbf{Математические абстракции}.
    \begin{enumerate}
        \item[$-$] \textbf{Точка} – хранит координаты x, y, z.
        \item[$-$] \textbf{Вектор} – хранит направление по x, y, z.
        \item[$-$] \textbf{Фигура} – хранит вершины, нормаль, цвет.
    \end{enumerate}
\item \textbf{Интерфейс} – используются библиотечные классы для предоставления доступа к интерфейсу.
\item \textbf{Графический обработчик} - абстрактная структура, выполянющая реализацию алгоритмов.
\item \textbf{Фабрики} для сцены, интерфейса, обработчика - для возможной подмены в ходе разработки или дополнения в дальнейшем.
\item \textbf{Композит} - объект, который будет содержать в себе другие объекты
\end{enumerate}
