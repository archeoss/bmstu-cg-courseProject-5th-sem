\specsection{ВВЕДЕНИЕ}
% \addcontentsline{toc}{specsection}{Введение}
Целью данного курсового проекта является разработка ПО, визуализирующие различные композиции тел вращения.
Это включает в себя такие вещи, как
\begin{enumerate}%[ 1{)}] 
    \item Воссоздание сложных композиций тел.
    \item Создание тел вращения путем вращения выше обозначенных композиций вокруг установленной оси.
    \item Представление полноценной освещенной сцены с композициями объектов.
    \item Обзор рабочей сцены из различных ее участков посредством нескольких камер.
    \item Интерактивность с различными объектами сцены, такими как:
        \begin{itemize}
        \item[$-$] Камеры;
        \item[$-$] Модели;
        \item[$-$] Композиты.
    \end{itemize}
\end{enumerate}
Для достижения поставленных целей необходимо решить следующие задачи:
\begin{itemize}
    \item[$-$] описать структуру трехмерной сцены;
    \item[$-$] реализовать оптимальные алгоритмы представления, преобразования и визуализации твердотельной модели;
    \item[$-$] реализовать аппаратную обработку всех элементов сцены;
    \item[$-$] спроектировать процесс моделирования сцены;
    \item[$-$] описать использующиеся структуры данных;
    \item[$-$] определить средства программной реализации;
    \item[$-$] провести экспериментальные замеры временных характеристик разработанного ПО.
\end{itemize}
В ходе курсовой работы будут затронуты такие темы, как:
\begin{itemize}
    \item[$-$] понятия о телах и поверхностях вращения;
    \item[$-$] удаление невидимых линий и поверхностей;
    \item[$-$] методы закрашивания объемных тел;
    \item[$-$] преимущества и особенности языка Rust.
\end{itemize}
