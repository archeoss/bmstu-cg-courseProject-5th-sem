\specsection{ЗАКЛЮЧЕНИЕ}
% \addcontentsline{toc}{specsection}{ЗАКЛЮЧЕНИЕ}
Целью данного курсового проекта была достигнута, то есть был разработан программный продукт, позволяющий создавать и редактировать композиции из трехмерных графических тел вращения. Также ПО предоставляет возможности настройки геометрических характеристик объектов, положения камеры и положения источника освещения. 

Для достижение цели были выполнены следующие задачи:
\begin{itemize}
	\item проведен анализ существующих алгоритмов компьютерной графики, использующие для создание реалистичной модели и трехмерной сцены;
	\item выбраны наиболее подходящие алгоритмы (алгоритмы удаления невидимых линий, методы закраски, модели освещения) для решения поставленной задачи;
	\item спроектированы архитектура и графический интерфейс программы;
	\item выбраны средства реализации программного обеспечения;
	\item разработано ПО и реализация выбранных алгоритмов и структур данных;
	\item проведены замеры временных характеристик разработанного программного обеспечения.  
\end{itemize}

В процессе исследовательской работы было выяснено, что полученная модель имеет линейную зависимость времени отрисовки от количества граней и количества объектов. Данное свойство говорит о том, что программа будет хорошо работать со средним количеством объектов, однако для большего их количества понадобятся дополнительные оптимизации вычислений.
В частности, быстродействие низкоуровневых частей программы, отвечающих за растеризацию полигонов, может быть улучшено за счет замены их программной реализации на поддерживаемую аппаратно из интерфейсов графических движков, например, OpenGL или DirectX.
