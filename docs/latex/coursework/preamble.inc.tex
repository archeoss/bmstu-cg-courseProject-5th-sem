\documentclass[ut8x, 14pt, oneside, a4paper]{extarticle}
\usepackage{extsizes}
\usepackage{cmap} % Улучшенный поиск русских слов в полученном pdf-файле
\usepackage[english,russian]{babel}
\usepackage{fontspec}
\setmainfont{Times New Roman}
\usepackage{misccorr}
\usepackage{indentfirst}
\usepackage{enumitem}
\setlength{\parindent}{1.25cm}
\usepackage{multirow}
\renewcommand{\baselinestretch}{1.5}
\setlist{nolistsep} % Отсутствие отступов между элементами \enumerate и \itemize

%\usepackage{pscyr} % Нормальные шрифты
\usepackage{amsmath}
\usepackage{geometry}
\geometry{left=30mm}
\geometry{right=15mm}
\geometry{top=20mm}
\geometry{bottom=20mm}
% \usepackage{titlesec}
% \titleformat{\section}
% {\normalsize\bfseries}
% {\thesection}
% {1em}{}
% \titlespacing*{\chapter}{0pt}{-30pt}{8pt}
% \titlespacing*{\section}{\parindent}{*4}{*4}
% \titlespacing*{\subsection}{\parindent}{*4}{*4}
% \usepackage{setspace}
% \onehalfspacing % Полуторный интервал
% \frenchspacing
% \usepackage{indentfirst} % Красная строка
% \usepackage{titlesec}
% \titleformat{\chapter}{\LARGE\bfseries}{\thechapter}{20pt}{\LARGE\bfseries}
% \titleformat{\section}{\Large\bfseries}{\thesection}{20pt}{\Large\bfseries}
\usepackage{listings}
\usepackage{listings-rust}
\usepackage{xcolor}
\usepackage{pdfpages}
\usepackage{enumerate}
\lstdefinestyle{rust}{
    language=Rust,
    backgroundcolor=\color{white},
    basicstyle=\footnotesize\ttfamily,
    keywordstyle=\color{purple},
    stringstyle=\color{green},
    commentstyle=\color{gray},
    numbers=left,
    stepnumber=1,
    numbersep=5pt,
    frame=single,
    tabsize=4,
    captionpos=t,
    breaklines=true,
    breakatwhitespace=true,
    escapeinside={\#*}{*)},
    morecomment=[l][\color{magenta}]{\#},
    columns=fullflexible
}
\usepackage{pgfplots}
\usetikzlibrary{datavisualization}
\usetikzlibrary{datavisualization.formats.functions}
\usepackage{graphicx}
\newcommand{\imgw}[3] {
    \begin{figure}[!ht]
        \center{\includegraphics[width=#1]{inc/img/#2}}
        \caption{#3}
        \label{img:#2}
    \end{figure}
}
\newcommand{\img}[3] {
    \begin{figure}[!ht]
        \center{\includegraphics[height=#1]{inc/img/#2}}
        \caption{#3}
        \label{img:#2}
    \end{figure}
}
\newcommand{\boximg}[3] {
    \begin{figure}[!ht]
        \center{\fbox{\includegraphics[height=#1]{assets/img/#2}}}
        \caption{#3}
        \label{img:#2}
    \end{figure}
}
\usepackage[justification=centering]{caption} % Настройка подписей float объектов
% \usepackage[unicode,pdftex]{hyperref} % Ссылки в pdf
% \hypersetup{hidelinks}
\newcommand{\code}[1]{\texttt{#1}}
\usepackage{icomma} % Интеллектуальные запятые для десятичных чисел
\usepackage{csvsimple}

% Переопределение стандартных \section, \subsection, \subsubsection по ГОСТу;
% Переопределение их отступов до и после для 1.5 интервала во всем документе

\usepackage{titlesec}
\usepackage{array}
\newenvironment{signstabular}[1][1]{
    \renewcommand*{\arraystretch}{#1}
    \tabular
    }{
    \endtabular
}


% \filcenter
% \titleformat{\section}[block]
% {\bfseries\normalsize}{\thesection}{1em}{}
\titlespacing{\section}{0mm}{0mm}{8mm}

\titleformat{\subsection}[hang]
{\bfseries\normalsize}{\thesubsection}{1em}{}
\titlespacing\subsection{\parindent}{12mm}{12mm}

\titleformat{\subsubsection}[hang]
{\bfseries\normalsize}{\thesubsubsection}{1em}{}
\titlespacing\subsubsection{\parindent}{12mm}{12mm}

\titleformat{name=\section}[block]
{\normalfont\normalsize\bfseries\hspace{\parindent}}
{\thesection}
{1em}
{}
\titleformat{name=\section,numberless}[block]
{\normalfont\normalsize\bfseries\centering}
{}
{0pt}
{}

\newcommand{\anonsection}[1]{%
    \section*{\centering#1}%
    \addcontentsline{toc}{section}{#1}%
}

\newcommand{\specsection}[1]{
    \section*{\centering#1}
    \addcontentsline{toc}{section}{#1}
}


